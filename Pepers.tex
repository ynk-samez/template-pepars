%%: 図
%\begin{figure}[H]
%	\centering
%	\includegraphics[width=90mm]{figure/eqc.eps}
%	\caption{簡易等価回路}
%	\label{fig:eqc}
%\end{figure}
\documentclass[10.5pt,a4j,dvipdfmx,openany]{jsbook}
\usepackage{amsmath}
\usepackage{subfigure}
\usepackage{overcite}
\renewcommand{\subfigtopskip}{5pt}	% 図の上の隙間。上図の副題と下図の間。
\renewcommand{\subfigbottomskip}{0pt} % 図の下の隙間。副題と本題の間。
\renewcommand{\subfigcapskip}{-6pt}	% 図と副題の間
\renewcommand{\subcapsize}{\scriptsize} % 副題の文字の大きさ

%以下は式番号をsection(節)単位で付けてくれるものです 
%例えば 1節の5番目の数式の場合は式1.5となります
\numberwithin{equation}{section}
\numberwithin{table}{section}
\numberwithin{figure}{section}
\usepackage{caption}
\usepackage{listings}
\usepackage{cases}%連立方程式とかの横の{を書くためのやつ。
\usepackage{amsfonts}
\usepackage{bm}%$\bm{アルファベット文字}$で太文字のベクトル表記ができます
\usepackage{eee}
\usepackage{url}%参考文献とかでurl使うときに。\url{https://~}
\usepackage{here}%図があっちこっちいかないようにできる
\usepackage{amsmath,amssymb}%数式関連
%表関連
\usepackage{subfigure}
\usepackage{comment}
\usepackage{multirow}
\usepackage{booktabs}
\usepackage{tabularx}
\usepackage{subfig}
\usepackage{paralist}
\usepackage{pdfpages}
\usepackage{titlesec}
\usepackage[deluxe]{otf}
\usepackage{wrapfig}
\usepackage{lettrine}
\usepackage{amsmath}
\usepackage{graphicx}
\usepackage{color}
\setlength{\fboxrule}{1pt}
\graphicspath{{./figs/}}
\usepackage{algpseudocode}
\usepackage{algorithm}
\usepackage{wallpaper}
\usepackage{listings,jvlisting} 
\renewcommand{\figurename}{Fig. }
\renewcommand{\tablename}{Tab. }
\makeatletter
\renewcommand{\ALG@name}{Algorithm.}

\renewcommand{\subfigtopskip}{5pt}	% 図の上の隙間。上図の副題と下図の間。
\renewcommand{\subfigbottomskip}{0pt} % 図の下の隙間。副題と本題の間。
\renewcommand{\subfigcapskip}{-6pt}	% 図と副題の間
\renewcommand{\subcapsize}{\scriptsize} % 副題の文字の大きさ

% 節以下の見出しがゴシック体になってしまっていたので、
% 太字明朝体になるように修正。
\renewcommand{\headfont}{\bfseries}

\makeatletter%% プリアンブルで再定義する際は必須

\def\@makechapterhead#1{\hbox{}%
  \vskip2\Cvs
  {\parindent\z@
%  \raggedright% 左揃え(オリジナルの定義)
  \centering% 中央揃え
%  \raggedleft% 右揃え
  \normalfont\huge\sffamily\gtfamily%% フォントを変更する
  \leavevmode
  \ifnum \c@secnumdepth >\m@ne
    \setlength\@tempdima{\linewidth}%
%%%    \if@mainmatter% ← report クラスの場合この行不要
%    \setbox\z@\hbox{\@chapapp\thechapter\@chappos\hskip1zw}%
%    \addtolength\@tempdima{-\wd\z@}%
%    \unhbox\z@\nobreak
%%%    \fi% ← report クラスの場合この行不要
    \vtop{\hsize\@tempdima\@chapapp\thechapter\@chappos\hskip1zw#1}%
  \else
    #1\relax
  \fi}\nobreak\vskip3\Cvs}
%
\def\@makeschapterhead#1{\hbox{}%
  \vskip2\Cvs
  {\parindent\z@
%  \raggedright% 左揃え(オリジナルの定義)
  \centering% 中央揃え
%  \raggedleft% 右揃え
  \normalfont\huge\sffamily\gtfamily%% フォントを変更する
  \leavevmode
  \setlength\@tempdima{\linewidth}%
  \vtop{\hsize\@tempdima#1}}\vskip3\Cvs}
\makeatother%% プリアンブルで再定義する際は必須

\makeatletter

%%% 見出しの体裁カスタマイズ

% 章見出しの体裁カスタマイズ
% *なし
\renewcommand{\@makechapterhead}[1]{% #1: 見出し
    \vspace*{-2\baselineskip}%
    \noindent\hrulefill\par%
    \vtop to 5\baselineskip{% 5行分の高さを確保
        \hspace*{\fill}\textbf{\large 第\thechapter 章}\par%
        \hspace*{\fill}\textbf{\raisebox{-5pt}{\LARGE #1}}\par%
    }%
    \noindent\hrulefill\par}
% *あり
\renewcommand{\@makeschapterhead}[1]{% #1: 見出し
    \vspace*{-2\baselineskip}%
    \noindent\hrulefill\par%
    \vtop to 5\baselineskip{% 5行分の高さを確保
        \hspace*{\fill}\textbf{\raisebox{-5pt}{\LARGE #1}}\par%
    }%
    \noindent\hrulefill\par}

\makeatother

\renewcommand{\headfont}{\bfseries}


\usepackage{algpseudocode}
\usepackage{algorithm}
\usepackage{wallpaper}
\usepackage{listings,jvlisting} 
\renewcommand{\figurename}{Fig. }
\renewcommand{\tablename}{Tab. }
\makeatletter
\renewcommand{\ALG@name}{Algorithm.}
\makeatother
\title{\LARGE\fbox{令和4年度 卒業審査論文}\\\mbox{}\\
  神経免疫相互作用に着想を得たマルチエージェントシステム型Neural-Networkの提案\\ \mbox{}\\
\textit{Proposal of Neural-Network composed of multi-agent system inspired by neuroimmune interaction}}
\author{柚木 開登}

\begin{document}
\maketitle
\includepdf[pages=-]{title.pdf}
% トップページを書く
\tableofcontents
\listoftables
\listoffigures
\clearpage
\section*{あらまし}
\chapter{序論}
\begin{comment}
\section{概要}
Neural-Networkによる大規模データセットを用いた機械学習は, 
創薬, 自動運転, 金融に至るまで成果を残しており, 
今後も期待される分野は多岐にわたる.
しかし従来, 大規模な機械学習には大量のGPUと高性能なプロセッサを必要とするため, 
導入は容易ではなかった.
こうした課題を解消し, 機械学習の市場導入を支えてきたのが
クラウドコンピューティングである.

クラウドコンピューティングは処理をデータセンター上で実行し, 
結果を利用者に提供することで
計算機環境に付随する経済的・物理的な制約を解消した.
また, その性質故に情報環境の迅速な整備を可能とし, 
業務の効率化を実現させ情報社会を発展させた.
このような公益的な側面の一方で, 
クラウドコンピューティングは, 
プラットフォーム企業による市場寡占,  
データプライバシーに係る課題の顕在化,  
企業--利用者間の情報の非対称性, 
データセンターの消費電力の増大といった問題を生じさせた.
適切な市場競争の回復, 情報の平等性, 省エネルギー社会の樹立, 
データプライバシーの保護に向けて, 
クラウドコンピューティングに代わる新しいコンピューティング
アーキテクチャであるエッジコンピューティングの技術開発が進められている.
従って, クラウドに依存してきた機械学習においても
エッジコンピューティングへの対応が求められる.

本章では, 国内の次世代型社会基盤と市場創出への期待, 
及び国際的なプライバシー保護の機運の高まりに焦点をあて, 
クラウドコンピューティング依存からの脱却に向けた
国内•国際的な動きについて紹介し, 
エッジコンピューティングにおける機械学習の将来に
ついて述べる.
  
\end{comment}
\section{社会的背景}
\label{背景}
%\lettrine[lines=2]%
\subsection*{第五期科学技術基本計画実現に向けて}
内閣府が提出した第五期科学技術基本計画\cite{第五期科学技術基本計画}において, 
仮想空間と現実空間が高度に融合したデータ駆動社会, 所謂Society5.0が提唱された.
Society5.0では, 
これまで個別に機能していた生産, 流通, 教育, 医療, 金融等, 
異分野のあらゆるシステムが仮想空間を通じて協調・統合することで, 
多様な知識・技術の融合が行われる. 
またこれにより, 産業構造の変革や技術革新を促し, 
社会の至るところで新たな価値が創出することが期待される.
このような持続的イノベーション創出基盤を確立することは, 
今後の我が国の国際競争における優位性を確保し, ひいては経済成長の強力な足場となり得る.

Society5.0の実現には, 
現実空間の潜在情報をデータ化するセンシング, 
安全性・信頼性が保証された通信プロトコル, 
高速かつ大容量なネットワークといった
情報の収集・蓄積・通信・解析に係る技術が基盤となる.

翻って, 現在の我が国の電気・情報・通信分野の実情を俯瞰すると, 
情報通信の最も基礎となる半導体加工技術及び精密部品製造については
世界に伍する技術を持つ一方で, 
ハードウェア以上の階層の行程, 
すなわちソフトウェア, OS, 仮想化技術に至るまで
大部分が外国製品に依存しており,  
またそれらが既に市場を寡占しているため, 
優れたサービスが開発されても, 市場圧力に負け撤退する状態が続いている.
さらに, 我が国が強みとするハードウェア技術に関しても対応が迫られている.
近年, アジアを中心に我が国からの技術・人材流出しており, 
同市場における競争力の低下が指摘されている.
また同時に, クラウド技術の発展により, 
クライアント側のハードウェアに依存せずにパフォーマンスを
提供することが可能になったことも受け, 
ハードウェア市場の長期的な成長率は減少しつつある.
ハードウェアを強みとする我が国において, 
同市場の需要の喪失は計り知れない.

こうした現状を無視して, 
情報技術が基盤となるSociety5.0への転換を進めた場合, 
従来持っていた市場を喪失するばかりか, 
外国への(特にソフトウェア領域を中心とした)技術依存性を高め, 
市場寡占を加速させ停滞した状況の打破がより難しくなる. 
従って, Society5.0への移行に際しては, 
単に転換を推進するだけでなく目下のIT市場の寡占状態を解消し,   
適切な市場競争を回復することによって我が国のソフトウェア産業の振興を促すことが望ましい.
\subsection*{データセンターとエネルギー問題と分散化}
データセンターは, 電子メールやWebサイトのサーバー, 
あるいはクラウドサービス用のストレージやクラスタ等の計算設備を集約設置する施設である.
その規模は, 情報社会の広まりに伴って年々増加しており, 現代においてデータセンターはインフラストラクチャの
バックボーンとしての位置を占める.

一方で, データセンターはその性質故に絶え間なく大量のトラフィックを
取り扱うため, 全世界の2%に及ぶ莫大な電力消費を行うことが知られており, 
こうした課題に対する取り組みとして, 例えば冷却効率の改善がある.
データセンターで消費される電力のうち
30〜50%が計算設備を冷却するために
用いられている. 
拠って冷却効率を改善することで, データセンターの大幅な省エネルギー化の実現が期待できる.

別のアプローチではエッジデータセンター等のエッジ分散によるトラフィックの抑制がある. 
エッジデータセンターは, 
データセンターを小規模化させ, 利用者に近い場所に設置することで
トラフィックの集中を防ぎ, データの遅延を最小化させるという考えによる次世代のデータセンターである.
従ってエッジデータセンターは, 従来のハイパースケールデータセンターと比較して, 
取り扱うトラフィックが減少し機器からの発熱量も小さくなるため, 冷却に係る電力を
少なく抑えることができる.

いずれにせよ, エネルギー問題に直面しつつある我々人類にとって, 
あらゆる機器の省エネルギー化は喫緊の課題であり, データセンターのあり方について変革が迫られている.
\subsection*{巨大IT企業の台頭, 国家安全保障, プライバシー保護}
2010年以降,  巨大IT企業が存在感を増している.
特に, Google, Apple, Facebook(Meta), Amazon, Microsoftは
俗にGAFAMと呼称され, その莫大な資本力故に市場を支配するだけにとどまらず, 
それぞれが世界各国の政治・経済に多大な影響力を有するに至っている.

GAFAMが数年の間に急速に成長し, 
今なお強大な影響力を持つ要因はさまざまである.
社会的要因を排除したひとつの視点は, 機械学習を中心としたユーザーデータの収集と解析である.
彼らは, 登録されたアカウント情報やWebページの訪問履歴, 商品の購入履歴, 
視聴した音楽, アシスタントAIへの質問, あるいはコンテンツの使用時刻等
から性別・年齢・信条・健康状態・職業・収入・趣味嗜好を含めた大量のデータを収集・解析し, 
市場需要の高い製品やサービスの開発に注力し, 生産性を向上させ競争に勝ってきた.

ユーザーデータの収集は製品やサービスの改善として顧客に還元される一方で, 
どこで, どのように, どこまで用いられるのかが見えにく, ユーザーデータ収集に対する懸念を呼び起こした.

IT市場が規模を拡大するに従って, 個人情報を巡るさまざまな議論が行われたが, 
しばらくの間, 規制当局は自由市場を確保しようとした.
各国がプライバシー保護に関する規制を加速させたのは, 
2013年に元アメリカ国家安全保障局(NSA)職員のEdward・J・Snowdenが行った内部告発の影響が大きい.
Snowdenが告発した内容は, 
「NSAが, 国内外のインターネット回線・電話回線を傍受し, アメリカの同盟国を含む世界中のあらゆる通信を監視・収集している」というものだった.
この大量監視プログラムは, PRISMと称され,
Google, Yahoo, Apple, FacebookといったIT企業が協力していた.

Snowdenのリークは, 巨大IT企業への不信感とアメリカへの非難を生んだ.
特に, EUは強い不快感を示し, 
NSAのプライバシー侵害に対する
非難決議を採択したほか 
EU-US Privacy Shieldの制定や
GDPR(General Data Protection Regulation)の議論の加速など
プライバシー保護のための法整備を進めた\cite{GDPRsnowden}.

とりわけ, GDPRは, 
EUという巨大な経済圏に適用される規制として, 世界に対する影響力が大きく, 
また, 後の国際的なプライバシー保護に関する動きの最初期に行われたものとして意義深い.

以下, GDPRを中心とした主なプライバシー保護法とその制定年を列挙する(\wtab{個人情報保護}). 
\begin{table}[H]
  \centering
  \caption{世界各国の個人情報保護規制}
  \label{tab:個人情報保護}
  \begin{tabular}{lll}
    \toprule
    適用年&規制&対象地域\\ \midrule
    2018年&GDPR; General Data Privacy Regulation&欧州連合加盟国\\
    2020年&CPRA; California Privacy Rights Act&アメリカ合衆国カリフォルニア州\\
    2022年&PDPR; Personal Data Protection Regulation&タイ王国\\
          &個人情報の保護に関する法律等の一部を改正する法律&日本\\
    2023年&CCPA; California Consumer Privacy Act&アメリカ合衆国カリフォルニア州\\
    \bottomrule
  \end{tabular}
\end{table}
以上のようなプライバシー保護の試みに関わらず,  
機械学習, 換言してプライバシーデータ収集の恩恵を得てきた
GAFAMら巨大IT企業はその莫大な資金力を持って規制を突破している現状があり\cite{GDPRの影響}, 
IT市場の寡占状態に対する影響は限定的である.
\clearpage
\section{エッジコンピューティングへの期待}
\subsection*{エッジコンピューティングが解決するクラウド社会の課題}
先に述べたように, 社会はクラウド志向なサービスの利便性を享受する一方で, 
プラットフォーマーによる市場寡占, データセンターでの電力消費増大, 
そしてデータプライバシーの保護といった課題に直面している.

クラウドの課題を克服する新しいアーキテクチャとして
エッジコンピューティング(Edge Computing)が提唱されている.
エッジコンピューティングでは, 
データや処理をデータセンターに集約せずに
利用者に物理的に近い計算機で完結するため, 
負荷がネットワーク上に分散されるとともに, 通信の機会の最小化を図ることができる.
また, 別の利点としてユーザーは環境内の計算資源を利用するため, 
特定のプラットフォームへの依存から脱却することを可能とし, 
市場構造の変革が促進され, 剰えその結果, 
プロットフォーム企業による市場寡占が弱まり適切な市場競争が回復することが期待できる.
\subsection*{遊休計算資源の利用とマルチエージェントシステム}
エッジ環境では, 計算資源の制約がクラウドのそれと比較して厳しくなる.
斯かれどエッジ環境内の計算資源はそのすべてが利用されていることは稀であり, 
大多数が遊休状態に入っている.
そこで, エッジ環境内の複数の遊休計算資源を利用して大規模な分散計算を行おうという試みがある.
IoTデバイスやゲーム機などの遊休計算資源を分散計算に投入することができれば, 
従来の設備を全く更新せずに, 大きな計算をエッジ環境内で完結することができるためである.
しかしながら, ここで課題が生じる.
ある一つのタスクを遊休計算資源に分散処理しようと考えるとき, これらは一般に異性能であるため,  
計算資源を最大限活用した分散計算を行うには, 適切な大きさのサブタスクの分割と
全体として計算資源の利用率が最大となるように配分しなければならない(\wfig{task}).
\begin{figure}[H]
  \centering
  \includegraphics[width=12cm]{task.pdf}
  \caption{巨大なタスクをいくつかの小タスクに分割し, エッジ環境内の遊休計算資源に割り当てる.}
  \label{fig:task}
\end{figure}
特に, 後者の「全体として計算資源の利用率が最大となるように配分する」ことは
NP完全な組合せ最適化問題として知られるナップサック問題そのものであり, 
解決するのは非常に難しい. 
この分野における研究としては, 
エージェント志向なシステムによるタスク分割最小化がある.

システムをマルチエージェントシステムとして構成することで, 
タスク分割の最小単位をエージェントとすることができる. 
またエージェントの増加・減少が自由にできるため与えられた計算環境に自動適応するようなモデルも
構築できる.
\subsection*{エッジコンピューティングにおける機械学習手法}
今日, 機械学習はあらゆる産業に投入され, それぞれにおいて多大な功績を残している.
機械学習に期待される分野は多岐に渡り, 今後も機械学習の導入は広まると予測される. 
エッジコンピューティングにおける機械学習として, 連合学習(Federareted-Learning:FL)が提案されている. 
連合学習では, 学習を各エッジ環境で行い, その結果得たパラメータ差分のみを通信し, 平均化することで
従来と同様な大規模なモデルを構築する.
\begin{figure}[H]
  \centering
  \includegraphics[width=15cm]{federareted-learning.pdf}
  \caption{協調型連合学習の概念図}
\end{figure}
連合学習ではプライバシーデータを含む学習データは各ノードに格納され, 
通信しないため保護される.
先に述べた国際的なデータプライバシー保護の機運の高まりから, 連合学習はエッジコンピューティング時代の
機械学習手法のスタンダードとなっていくだろう.
\section{本研究の目的}
連合学習の実用化に際しては, 
従来, データセンターで動いていたモデルを計算資源に乏しいエッジ環境に投入する必要があることから
少なくとも以下2点の課題が考えられる.
1つ目は学習パラメータ削減によるモデルの軽量化, 
2つ目は, エッジ環境内の不均一計算資源の利用
である.
前者については刈り込みや, ニューラルアーキテクチャ探索, 後者についてはエージェント志向アーキテクチャによる
分散処理システムや遊休計算資源を用いた大規模分散計算等があり, 
今日までに多様な先行事例・研究が成されている. 
両者を満たしたモデルを構築できれば, 
大規模なNeural-Networkのエッジ環境への投入を可能にし, 
かつ環境内の計算資源の利用率を最大化することが期待できるが,  
両者を満たしたモデルに関する研究は著者の知る限り存在しない.

そこで本研究では, 将来的に導入が期待されるエッジコンピューティングアーキテクチャにおける機械学習の
実用化に向けて, 自律分散可能なNeural-Networkのパラメータ削減手法を提案する.
提案にあたって, 生体の脳で行われる自律分散的なパラメータ削減であるグリア細胞による
シナプスの刈り込みによる神経回路の醸成に着目し, 関係する細胞及び構造をNeuro-Agent, Synapse-Agent, 
Glia-Agentという三種のエージェントとしてモデル化した. 
これにより, システム全体がマルチエージェントシステムとして構成されるため, 自律分散性が保障される.
\section{本論文の構成}
本論文では神経免疫相互作用に着想を得たマルチエージェントシステム型ニューラルネットワークについての提案を行う.
 本論文の以下の構成は次のようになっている。
 第2章では、本論文の前提となる先行事例・研究を紹介する.
 第3章では、提案アルゴリズムに関連する生物学的背景を紹介する.
 第4章では、提案アルゴリズムを提案する.
 第5章では、計算機実験の結果及び考察をする.
 第6章では結論を述べる.
 なお, 付録として本提案アルゴリズムの実装ソースコードを付してある.
\chapter{関連研究}
\chapter{謝辞}

\begin{thebibliography}{9}
%参考文献の書式。
%書籍:       著者,書名[,シリーズ名],出版社[,出版地],発行年[,ページ].
%論文:       筆者,表題,雑誌名{,巻号},発行年月{,ページ}.
%学会発表:   筆者,表題,会議名{,セッション名},開催年月日.
%新聞記事:   {筆者,}見出し,紙名[(夕刊 ],年月日[,ページ].
%Web ページ: {著者,}表題{,シリーズ名},サイト名,掲載社{, 作成日}, 閲覧日, URL
%個人的対話: 話者[,状況記述],年月日.
\bibitem{cloud_market}
総務省, 情報通信分野の現状と課題, \url{https://www.soumu.go.jp/johotsusintokei/whitepaper/ja/r04/html/nd236800.html}, 閲覧日:令和5年1月21日
\bibitem{MemberShip}
Shokri, Reza, et al. "Membership inference attacks against machine learning models." 2017 IEEE symposium on security and privacy (SP). IEEE, 2017.
\bibitem{第五期科学技術基本計画}
内閣府, 第五期科学技術基本計画, \url{https://www8.cao.go.jp/cstp/kihonkeikaku/5honbun.pdf},平成28年1月22日
\bibitem{Beyond5G}
五十嵐大和. ``Beyond5G 推進戦略 6G へのロードマップ.'' IEICE Conferences Archives. The Institute of Electronics, Information and Communication Engineers, 2020. 
\bibitem{BigTech}
Bremmer, Ian. "The Technopolar Moment: How Digital Powers Will Reshape the Global Order." Foreign Aff. 100 (2021): 112.
\bibitem{PRISM}
宮下紘. "個人情報保護とサイバー・セキュリティ: デジタル時代の双子." 比較法雑誌 49.4 (2016): 81-93.
\bibitem{GDPRsnowden}
Coyne, Hallie. "The untold story of Edward Snowden’s impact on the GDPR." The Cyber Defense Review 4.2 (2019): 65-80.
\bibitem{民主主義と自律分散}
原田泉. "米中新冷戦期の DX 推進と我が国独自のネットワーキング社会の実現." 危機管理研究 29 (2021): 1-14.
\bibitem{マルチエージェントシステムの基礎と応用}
大内 東, 山本雅人, 川村秀憲, "マルチエージェントシステムの基礎と応用 -- 複雑系工学の計算パラダイム --", コロナ社, 2003年
\bibitem{GDPRの影響}
野村総合研究所, プライバシーガバナンスの時代 ―法改正とGAFA規制に向けたプライバシー投資のあり方― \url{https://www.nri.com/-/media/Corporate/jp/Files/PDF/knowledge/report/cc/mediaforum/2022/forum338.pdf?la=ja-JP&hash=3C2C0B83004EFDBBC63F4650EE36D6F92B18966C}
\bibitem{GliaCell}
Types of Glia, lumen, Biology for Majors II,
\url{\url{https://courses.lumenlearning.com/wm-biology2/chapter/glial-cells/}}
\bibitem{GliaAssembly}
文部科学省, 新学術領域研究「グリアアセンブリによる脳機能発現の制御と病態」,\url{http://square.umin.ac.jp/glialassembl/research/index.html#con02}
\end{thebibliography}
\chapter{付録}

\end{document}




