%%: 図
%\begin{figure}[H]
%	\centering
%	\includegraphics[width=90mm]{figure/eqc.eps}
%	\caption{簡易等価回路}
%	\label{fig:eqc}
%\end{figure}
\documentclass[10.5pt,a4j,dvipdfmx,openany]{jsbook}
\usepackage{amsmath}
\usepackage{subfigure}
\usepackage{overcite}
\renewcommand{\subfigtopskip}{5pt}	% 図の上の隙間。上図の副題と下図の間。
\renewcommand{\subfigbottomskip}{0pt} % 図の下の隙間。副題と本題の間。
\renewcommand{\subfigcapskip}{-6pt}	% 図と副題の間
\renewcommand{\subcapsize}{\scriptsize} % 副題の文字の大きさ

%以下は式番号をsection(節)単位で付けてくれるものです 
%例えば 1節の5番目の数式の場合は式1.5となります
\numberwithin{equation}{section}
\numberwithin{table}{section}
\numberwithin{figure}{section}
\usepackage{caption}
\usepackage{listings}
\usepackage{cases}%連立方程式とかの横の{を書くためのやつ。
\usepackage{amsfonts}
\usepackage{bm}%$\bm{アルファベット文字}$で太文字のベクトル表記ができます
\usepackage{eee}
\usepackage{url}%参考文献とかでurl使うときに。\url{https://~}
\usepackage{here}%図があっちこっちいかないようにできる
\usepackage{amsmath,amssymb}%数式関連
%表関連
\usepackage{subfigure}
\usepackage{comment}
\usepackage{multirow}
\usepackage{booktabs}
\usepackage{tabularx}
\usepackage{subfig}
\usepackage{paralist}
\usepackage{pdfpages}
\usepackage{titlesec}
\usepackage[deluxe]{otf}
\usepackage{wrapfig}
\usepackage{lettrine}
\usepackage{amsmath}
\usepackage{graphicx}
\usepackage{color}
\setlength{\fboxrule}{1pt}
\graphicspath{{./figs/}}
\usepackage{algpseudocode}
\usepackage{algorithm}
\usepackage{algorithmic}
\usepackage{wallpaper}
\usepackage{listings,jvlisting} 
\renewcommand{\figurename}{Fig. }
\renewcommand{\tablename}{Tab. }
\makeatletter
\renewcommand{\ALG@name}{Algorithm.}

\renewcommand{\subfigtopskip}{5pt}	% 図の上の隙間。上図の副題と下図の間。
\renewcommand{\subfigbottomskip}{0pt} % 図の下の隙間。副題と本題の間。
\renewcommand{\subfigcapskip}{-6pt}	% 図と副題の間
\renewcommand{\subcapsize}{\scriptsize} % 副題の文字の大きさ

% 節以下の見出しがゴシック体になってしまっていたので、
% 太字明朝体になるように修正。
\renewcommand{\headfont}{\bfseries}

\makeatletter%% プリアンブルで再定義する際は必須

\def\@makechapterhead#1{\hbox{}%
  \vskip2\Cvs
  {\parindent\z@
%  \raggedright% 左揃え(オリジナルの定義)
  \centering% 中央揃え
%  \raggedleft% 右揃え
  \normalfont\huge\sffamily\gtfamily%% フォントを変更する
  \leavevmode
  \ifnum \c@secnumdepth >\m@ne
    \setlength\@tempdima{\linewidth}%
%%%    \if@mainmatter% ← report クラスの場合この行不要
%    \setbox\z@\hbox{\@chapapp\thechapter\@chappos\hskip1zw}%
%    \addtolength\@tempdima{-\wd\z@}%
%    \unhbox\z@\nobreak
%%%    \fi% ← report クラスの場合この行不要
    \vtop{\hsize\@tempdima\@chapapp\thechapter\@chappos\hskip1zw#1}%
  \else
    #1\relax
  \fi}\nobreak\vskip3\Cvs}
%
\def\@makeschapterhead#1{\hbox{}%
  \vskip2\Cvs
  {\parindent\z@
%  \raggedright% 左揃え(オリジナルの定義)
  \centering% 中央揃え
%  \raggedleft% 右揃え
  \normalfont\huge\sffamily\gtfamily%% フォントを変更する
  \leavevmode
  \setlength\@tempdima{\linewidth}%
  \vtop{\hsize\@tempdima#1}}\vskip3\Cvs}
\makeatother%% プリアンブルで再定義する際は必須

\makeatletter

%%% 見出しの体裁カスタマイズ

% 章見出しの体裁カスタマイズ
% *なし
\renewcommand{\@makechapterhead}[1]{% #1: 見出し
    \vspace*{-2\baselineskip}%
    \noindent\hrulefill\par%
    \vtop to 5\baselineskip{% 5行分の高さを確保
        \hspace*{\fill}\textbf{\large 第\thechapter 章}\par%
        \hspace*{\fill}\textbf{\raisebox{-5pt}{\LARGE #1}}\par%
    }%
    \noindent\hrulefill\par}
% *あり
\renewcommand{\@makeschapterhead}[1]{% #1: 見出し
    \vspace*{-2\baselineskip}%
    \noindent\hrulefill\par%
    \vtop to 5\baselineskip{% 5行分の高さを確保
        \hspace*{\fill}\textbf{\raisebox{-5pt}{\LARGE #1}}\par%
    }%
    \noindent\hrulefill\par}

\makeatother

\renewcommand{\headfont}{\bfseries}



\title{\LARGE\fbox{令和4年度 卒業論文}\\\mbox{}\\
  衛星間ネットワークによるスペースデブリの軌道推定技術の開発\\ \mbox{}\\
Development of space debris orbit estimation techniques\\using inter-satellite networks.}
\author{産技 権左衛門}

\begin{document}
\maketitle
% トップページを書く
\tableofcontents
\listoftables
\listoffigures
\clearpage
\chapter{序論}
\section{社会的背景}
\label{背景}
%\lettrine[lines=2]%
\subsection*{第五期科学技術基本計画実現に向けて}
内閣府が提出した第五期科学技術基本計画\cite{第五期科学技術基本計画}において, 
仮想空間と現実空間が高度に融合したデータ駆動社会, 所謂Society5.0が提唱された.
Society5.0では, 
これまで個別に機能していた生産, 流通, 教育, 医療, 金融等, 
異分野のあらゆるシステムが仮想空間を通じて協調・統合することで, 
多様な知識・技術の融合が行われる. 
またこれにより, 産業構造の変革や技術革新を促し, 
社会の至るところで新たな価値が創出することが期待される.
このような持続的イノベーション創出基盤を確立することは, 
今後の我が国の国際競争における優位性を確保し, ひいては経済成長の強力な足場となり得る.

翻って, ........
\subsection*{スペースデブリと宇宙開発}
現在, スペースデブリの推定個数は.....
\section{本研究の目的}
本研究では〜
\section{本論文の構成}
本論文ではスペースデブリ監視を目的とした衛星ネットワークを提案し,.....


\begin{thebibliography}{9}
%参考文献の書式。
%書籍:       著者,書名[,シリーズ名],出版社[,出版地],発行年[,ページ].
%論文:       筆者,表題,雑誌名{,巻号},発行年月{,ページ}.
%学会発表:   筆者,表題,会議名{,セッション名},開催年月日.
%新聞記事:   {筆者,}見出し,紙名[(夕刊 ],年月日[,ページ].
%Web ページ: {著者,}表題{,シリーズ名},サイト名,掲載社{, 作成日}, 閲覧日, URL
%個人的対話: 話者[,状況記述],年月日.
\bibitem{cloud_market}
総務省, 情報通信分野の現状と課題, \url{https://www.soumu.go.jp/johotsusintokei/whitepaper/ja/r04/html/nd236800.html}, 閲覧日:令和5年1月21日
\bibitem{MemberShip}
Shokri, Reza, et al. "Membership inference attacks against machine learning models." 2017 IEEE symposium on security and privacy (SP). IEEE, 2017.
\bibitem{第五期科学技術基本計画}
内閣府, 第五期科学技術基本計画, \url{https://www8.cao.go.jp/cstp/kihonkeikaku/5honbun.pdf},平成28年1月22日
\bibitem{Beyond5G}
五十嵐大和. ``Beyond5G 推進戦略 6G へのロードマップ.'' IEICE Conferences Archives. The Institute of Electronics, Information and Communication Engineers, 2020. 
\bibitem{BigTech}
Bremmer, Ian. "The Technopolar Moment: How Digital Powers Will Reshape the Global Order." Foreign Aff. 100 (2021): 112.
\bibitem{PRISM}
宮下紘. "個人情報保護とサイバー・セキュリティ: デジタル時代の双子." 比較法雑誌 49.4 (2016): 81-93.
\bibitem{GDPRsnowden}
Coyne, Hallie. "The untold story of Edward Snowden’s impact on the GDPR." The Cyber Defense Review 4.2 (2019): 65-80.
\bibitem{民主主義と自律分散}
原田泉. "米中新冷戦期の DX 推進と我が国独自のネットワーキング社会の実現." 危機管理研究 29 (2021): 1-14.
\bibitem{マルチエージェントシステムの基礎と応用}
大内 東, 山本雅人, 川村秀憲, "マルチエージェントシステムの基礎と応用 -- 複雑系工学の計算パラダイム --", コロナ社, 2003年
\bibitem{GDPRの影響}
野村総合研究所, プライバシーガバナンスの時代 ―法改正とGAFA規制に向けたプライバシー投資のあり方― \url{https://www.nri.com/-/media/Corporate/jp/Files/PDF/knowledge/report/cc/mediaforum/2022/forum338.pdf?la=ja-JP&hash=3C2C0B83004EFDBBC63F4650EE36D6F92B18966C}
\bibitem{GliaCell}
Types of Glia, lumen, Biology for Majors II,
\url{\url{https://courses.lumenlearning.com/wm-biology2/chapter/glial-cells/}}
\bibitem{GliaAssembly}
文部科学省, 新学術領域研究「グリアアセンブリによる脳機能発現の制御と病態」,\url{http://square.umin.ac.jp/glialassembl/research/index.html#con02}
\end{thebibliography}

\end{document}




