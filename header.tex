\usepackage{amsmath}
\usepackage{subfigure}
\usepackage{overcite}
\renewcommand{\subfigtopskip}{5pt}	% 図の上の隙間。上図の副題と下図の間。
\renewcommand{\subfigbottomskip}{0pt} % 図の下の隙間。副題と本題の間。
\renewcommand{\subfigcapskip}{-6pt}	% 図と副題の間
\renewcommand{\subcapsize}{\scriptsize} % 副題の文字の大きさ

%以下は式番号をsection(節)単位で付けてくれるものです 
%例えば 1節の5番目の数式の場合は式1.5となります
\numberwithin{equation}{section}
\numberwithin{table}{section}
\numberwithin{figure}{section}
\usepackage{caption}
\usepackage{listings}
\usepackage{cases}%連立方程式とかの横の{を書くためのやつ。
\usepackage{amsfonts}
\usepackage{bm}%$\bm{アルファベット文字}$で太文字のベクトル表記ができます
\usepackage{eee}
\usepackage{url}%参考文献とかでurl使うときに。\url{https://~}
\usepackage{here}%図があっちこっちいかないようにできる
\usepackage{amsmath,amssymb}%数式関連
%表関連
\usepackage{subfigure}
\usepackage{comment}
\usepackage{multirow}
\usepackage{booktabs}
\usepackage{tabularx}
\usepackage{subfig}
\usepackage{paralist}
\usepackage{pdfpages}
\usepackage{titlesec}
\usepackage[deluxe]{otf}
\usepackage{wrapfig}
\usepackage{lettrine}
\usepackage{amsmath}
\usepackage{graphicx}
\usepackage{color}
\setlength{\fboxrule}{1pt}
\graphicspath{{./figs/}}
\usepackage{algpseudocode}
\usepackage{algorithm}
\usepackage{algorithmic}
\usepackage{wallpaper}
\usepackage{listings,jvlisting} 
\renewcommand{\figurename}{Fig. }
\renewcommand{\tablename}{Tab. }
\makeatletter
\renewcommand{\ALG@name}{Algorithm.}

\renewcommand{\subfigtopskip}{5pt}	% 図の上の隙間。上図の副題と下図の間。
\renewcommand{\subfigbottomskip}{0pt} % 図の下の隙間。副題と本題の間。
\renewcommand{\subfigcapskip}{-6pt}	% 図と副題の間
\renewcommand{\subcapsize}{\scriptsize} % 副題の文字の大きさ

% 節以下の見出しがゴシック体になってしまっていたので、
% 太字明朝体になるように修正。
\renewcommand{\headfont}{\bfseries}

\makeatletter%% プリアンブルで再定義する際は必須

\def\@makechapterhead#1{\hbox{}%
  \vskip2\Cvs
  {\parindent\z@
%  \raggedright% 左揃え(オリジナルの定義)
  \centering% 中央揃え
%  \raggedleft% 右揃え
  \normalfont\huge\sffamily\gtfamily%% フォントを変更する
  \leavevmode
  \ifnum \c@secnumdepth >\m@ne
    \setlength\@tempdima{\linewidth}%
%%%    \if@mainmatter% ← report クラスの場合この行不要
%    \setbox\z@\hbox{\@chapapp\thechapter\@chappos\hskip1zw}%
%    \addtolength\@tempdima{-\wd\z@}%
%    \unhbox\z@\nobreak
%%%    \fi% ← report クラスの場合この行不要
    \vtop{\hsize\@tempdima\@chapapp\thechapter\@chappos\hskip1zw#1}%
  \else
    #1\relax
  \fi}\nobreak\vskip3\Cvs}
%
\def\@makeschapterhead#1{\hbox{}%
  \vskip2\Cvs
  {\parindent\z@
%  \raggedright% 左揃え(オリジナルの定義)
  \centering% 中央揃え
%  \raggedleft% 右揃え
  \normalfont\huge\sffamily\gtfamily%% フォントを変更する
  \leavevmode
  \setlength\@tempdima{\linewidth}%
  \vtop{\hsize\@tempdima#1}}\vskip3\Cvs}
\makeatother%% プリアンブルで再定義する際は必須

\makeatletter

%%% 見出しの体裁カスタマイズ

% 章見出しの体裁カスタマイズ
% *なし
\renewcommand{\@makechapterhead}[1]{% #1: 見出し
    \vspace*{-2\baselineskip}%
    \noindent\hrulefill\par%
    \vtop to 5\baselineskip{% 5行分の高さを確保
        \hspace*{\fill}\textbf{\large 第\thechapter 章}\par%
        \hspace*{\fill}\textbf{\raisebox{-5pt}{\LARGE #1}}\par%
    }%
    \noindent\hrulefill\par}
% *あり
\renewcommand{\@makeschapterhead}[1]{% #1: 見出し
    \vspace*{-2\baselineskip}%
    \noindent\hrulefill\par%
    \vtop to 5\baselineskip{% 5行分の高さを確保
        \hspace*{\fill}\textbf{\raisebox{-5pt}{\LARGE #1}}\par%
    }%
    \noindent\hrulefill\par}

\makeatother

\renewcommand{\headfont}{\bfseries}

